\documentclass{article}
\usepackage[utf8]{inputenc}
\usepackage[T2A]{fontenc}
\usepackage[russian]{babel}
\usepackage[T1]{fontenc}

\title{BrickGame v1.0}
\author{roostaha}
\date{November 2024}

\begin{document}

\maketitle

\section{Вступление}
Программа разработана на языке C стандарта C11 с использованием компилятора gcc. Визуальная часть реализована с помощью библиотеки \textbf{ncurses}. Игровое поле представляет собой сетку размером десять «пикселей» в ширину и двадцать «пикселей» в высоту, что соответствует стандартным размерам игрового поля в консольных версиях игры.

\section{Управление}
Управление осуществляется через следующие клавиши:
\begin{itemize}
    \item 'ENTER' - Начало игры.
    \item 'P' - Пауза.
    \item 'ESC' - Завершение игры.
    \item Стрелка влево — движение фигуры влево.
    \item Стрелка вправо — движение фигуры вправо.
    \item Стрелка вниз — ускоренное падение фигуры.
    \item Стрелка вверх — не используется в данной версии игры.
    \item 'Z' - Вращение фигуры.
\end{itemize}

\section{Конечный автомат}
Логика игры построена на конечном автомате, который описывает следующие состояния:
\begin{itemize}
    \item \textbf{StartGame} - игра находится в ожидании нажатия клавиши 'ENTER' для начала.
    \item \textbf{Spawn} - в этом состоянии создается новый блок и выбирается следующий для последующего спавна.
    \item \textbf{Moving} - активное состояние игры, в котором обрабатывается пользовательский ввод (перемещение фигуры, поворот, падение).
    \item \textbf{Shifting} - состояние активируется при завершении таймера. В этом режиме текущий блок опускается на одну линию вниз. Если столкновения не произошло, игра возвращается в состояние \textbf{Moving}; в противном случае переходит в состояние \textbf{Attaching}.
    \item \textbf{Attaching} - активируется, когда блок достигает нижней части поля или сталкивается с другими блоками. Если в этом состоянии образуются полностью заполненные линии, они удаляются, а игра переходит в состояние \textbf{Spawn}. Если блок достигает верхней границы, игра возвращается в состояние \textbf{StartGame}.
    \item \textbf{End} - при нажатии клавиши 'ESC' игра завершается.
\end{itemize}

\end{document}
